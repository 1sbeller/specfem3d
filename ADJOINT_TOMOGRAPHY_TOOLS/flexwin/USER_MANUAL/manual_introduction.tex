\chapter{Introduction}

The FLEXWIN software package automates the time-window selection problem for
seismologists.  It operates on pairs of observed and synthetic single component
seismograms, defining windows that cover as much of a given seismogram as
possible, while avoiding portions of the waveform that are dominated by noise.  

FLEXWIN selects time windows on the synthetic seismogram within which the waveform
contains a distinct energy arrival, then requires an adequate correspondence
between observed and synthetic waveforms within these windows.

There is no restriction on the type of simulation used to
generate the synthetics. Realistic Earth models and more complete wave
propagation theories yield waveforms that are more similar to the observed
seismograms, and thereby promote the selection of measurement windows covering
more of the available data.  The input seismograms can be measures of
displacement, velocity or acceleration.  There is no requirement
for horizontal signals to be rotated into radial and transverse directions.

FLEXWIN is a configurable data selection process that can be adapted to
different tomographic scenarios by tuning a handful of parameters.  Although
the algorithm was designed for use in 3D-3D adjoint tomography, its inherent
flexibility should make it useful in many data-selection applications.

For a detailed introduction to FLEXWIN as applied to seismic tomography, please consult \cite{MaggiEtal2009}.  If you use FLEXWIN for your own research, please cite \cite{MaggiEtal2009}.


