\documentclass[11pt]{article}
%
%\usepackage{kmath,kerkis}
\usepackage{mathpazo}
%\usepackage{palatino}
\usepackage[T1]{fontenc}
\usepackage{amsfonts}
\usepackage{mathrsfs}
\usepackage{amsthm}
\usepackage{dsfont}
%\usepackage{marvosym}
\usepackage{eurosym}
\usepackage{stmaryrd}
%\usepackage{gensymb}
\usepackage{hhline}
\usepackage{pifont}
\usepackage{fancybox}
\usepackage{enumerate}
%\usepackage{enumitem}
\usepackage{multirow}
\usepackage[table]{xcolor}
\usepackage[normalem]{ulem}
\usepackage{amsmath,amssymb}             % AMS Math
\usepackage[french]{babel}
\frenchbsetup{StandardLists=true}
%\usepackage{esint}
%\usepackage[latin1]{inputenc}

%\usepackage[hmargin=2cm,vmargin=4cm]{geometry}
%\usepackage[left=2.7cm,right=2cm,top=2.3cm,bottom=2.3cm,includefoot,includehead,headheight=13.6pt]{geometry}

%%%%%%%%%%%%%%%%%%%%%%%%%%%%%%%%%%%%%%%%%%%%%%%%%%%%%%

\definecolor{goldp}{rgb}{0.99 0.92 0.8}
\definecolor{oxblue}{RGB}{2 33 71}
\definecolor{cobalt}{RGB}{0 71 171}
\definecolor{impb}{RGB}{20 35 149}
\definecolor{graylight}{gray}{0.85}
\definecolor{denim}{RGB}{21 96 189}
\definecolor{bgreen}{RGB}{13 152 186}
\definecolor{bdeaux}{RGB}{153 12 20}
\definecolor{linkcol}{rgb}{0,0,0.4}
\definecolor{citecol}{rgb}{0.5,0.01,0.02}
\definecolor{lsb}{RGB}{120, 200, 200}

\DeclareSymbolFont{calletters}{OMS}{cmsy}{m}{n}
\DeclareSymbolFontAlphabet{\mathcal}{calletters}

\newcommand{\bw}{\mathbf{w}}
\newcommand{\be}{\mathbf{e}}
\newcommand{\bh}{\mathbf{h}}
\newcommand{\bT}{\mathbf{T}}
\newcommand{\bU}{\mathbf{U}}
\newcommand{\bV}{\mathbf{V}}
\newcommand{\bB}{\mathbf{B}}
\newcommand{\bD}{\mathbf{D}}
\newcommand{\bE}{\mathbf{E}}
\newcommand{\bH}{\mathbf{H}}
\newcommand{\bJ}{\mathbf{J}}
\newcommand{\bS}{\mathbf{S}}
\newcommand{\bW}{\mathbf{W}}
\newcommand{\bX}{\mathbf{X}}
\newcommand{\bY}{\mathbf{Y}}
\newcommand{\bn}{\mathbf{n}}
\newcommand{\bx}{\mathbf{x}}
\newcommand{\rmT}{\mathrm{T}}
\newcommand{\rmS}{\mathrm{S}}
\newcommand{\ci}{\mathbf{i}}
\newcommand{\ds}{\displaystyle}
\newcommand{\tr}{^{\mathrm{T}}}
\newcommand{\ie}{i.e.~}

\usepackage{hyperref}
\usepackage{amsfonts}
\usepackage{mathrsfs}
\usepackage{amssymb}
\usepackage{dsfont}
\usepackage{color}
%\usepackage[left=2.7cm,right=2cm,top=2.3cm,bottom=2.3cm,includefoot,includehead,headheight=13.6pt]{geometry}
\usepackage[hmargin=2.44cm,vmargin=2.2cm]{geometry}
\usepackage{graphics}
\usepackage{graphicx}

%
\def\R{\hbox{I\kern-0.2em\hbox{R}}}
\def\expg#1#2{{\vphantom{#1}}^{#2}\!#1}
\def\t#1{\expg{\!#1}{t}}

\def\restriction#1#2{\mathchoice
              {\setbox1\hbox{${\displaystyle #1}_{\scriptstyle #2}$}
              \restrictionaux{#1}{#2}}
              {\setbox1\hbox{${\textstyle #1}_{\scriptstyle #2}$}
              \restrictionaux{#1}{#2}}
              {\setbox1\hbox{${\scriptstyle #1}_{\scriptscriptstyle #2}$}
              \restrictionaux{#1}{#2}}
              {\setbox1\hbox{${\scriptscriptstyle #1}_{\scriptscriptstyle #2}$}
              \restrictionaux{#1}{#2}}}
\def\restrictionaux#1#2{{#1\,\smash{\vrule height .8\ht1 depth .85\dp1}}_{\,#2}}

\date{}

\begin{document}
{\huge {\sc {\centerline{Coupling AxiSEM and Specfem3D}} }}

\bigskip
\bigskip
\bigskip

\noindent {\bf{Authors:}} V. Monteiller, C. Durochat

\bigskip
\bigskip

\noindent NOTE : We have to finally choose which AxiSEM version we add and definitely use for coupling with SPECFEM3D.

\medskip

\noindent The paths present here are form specfem3d home directory.

\medskip

\noindent {\LARGE Here, \texttt{'run'}  is an alias for Curie defined as : \texttt{run='ccc\_msub -q standard'}.}

\bigskip

\section{STEP 1 : meshfem3D}

\begin{itemize}

\item[\textbullet] Go in 'EXAMPLES/coupling\_with\_EXTERNAL\_CODES/with\_AxiSEM/example\_1st\_for\_validation',

\smallskip

(OR, create another subdirectory in '[...]/with\_AxiSEM', copy files from '[...]/example\_1st\_for\_validation' and modify them if you need)

\item[\textbullet] {\color{red} CAUTION !! : Check the file 'paths\_for\_coupling\_SPECFEM3D\_AxiSEM.sh' and modify the variable 'HOME\_SPECFEM3D'} $\Longrightarrow$ this is the only path you have to modify for specfem3d

\item[\textbullet] Type : \\

\noindent ./clean\_this\_example\_dir.sh

\smallskip

\noindent run ./batch\_coupling\_step1\_create\_dir\_and\_paths\_+\_meshfem3d\_for\_AxiSEM.sh

\end{itemize}

\noindent It generate mesh files (on 1 proc on Curie), and copy files 'list\_gll\_boundary\_spherical.txt' and 'list\_gll\_boundary\_cartesian.txt' in AxiSEM solver dir.

\bigskip
\medskip

\noindent Also, check Meshfem3D with parameters $\Longrightarrow$ In the directory MESH/ :

\begin{itemize}

\item[\textbullet] 'ParFileMeshChunk'  : parameters for the chunk meshing

\item[\textbullet] 'iasp91\_dsm' or 'prem\_dsm' : background model used in both AxiSEM ans Specfem3D

\end{itemize}

\bigskip

\noindent In the Directory DATA/ :

\begin{itemize}

\item[\textbullet] 'CMTSOLUTION'  : with 0 in order to not use source inside

\item[\textbullet] 'coeff\_poly\_deg12' : to generate smooth intitial solution

\item[\textbullet] 'STATIONS'  : stations files

\item[\textbullet] Set the file 'Par\_file' with these parmeters :

\begin{itemize}
\item[\textbullet] SIMULATION\_TYPE=1
\item[\textbullet] SAVE\_FORWARD = .false.
\item[\textbullet] COUPLE\_WITH\_EXTERNAL\_CODE = .true
\item[\textbullet] EXTERNAL\_CODE\_TYPE    = 2
\end{itemize}

\end{itemize}


\section{STEP 2 : AxiSEM mesher}

\noindent Go in  : {\scriptsize  'utils/EXTERNAL\_CODES\_coupled\_with\_SPECFEM3D/AxiSEM\_for\_SPECFEM3D/AxiSEM\_modif\_for\_coupling\_with\_specfem/MESHER'}.

\medskip

\noindent Check the file 'inparam\_mesh'

\medskip

\noindent Type : ./submit.csh

\medskip

\noindent It run the mesher in serial : check file OUTPUT (wait for "....DONE WITH MESHER" to appear in OUTPUT)

\medskip

\noindent Type : ./movemesh.csh NAME\_OF\_DESTINATION\_MESH\_DIR

\medskip

\noindent It moves mesh files to ../SOLVER/MESHES/NAME\_OF\_DESTINATION\_MESH\_DIR

\medskip

\noindent For more details, see Axisem manual.

\section{STEP 3 : AxiSEM solver}

\noindent 2 files produced by meshfem3D (input\_box.txt and input\_box\_sem\_cart.txt) were copied (and the first line indicating the number of line to be read were add at the beginning of the files too) during the STEP 1 in the AxiSEM SOLVER dir.

\medskip

\noindent Go in {\scriptsize 'utils/EXTERNAL\_CODES\_coupled\_with\_SPECFEM3D/AxiSEM\_for\_SPECFEM3D/AxiSEM\_modif\_for\_coupling\_with\_specfem/SOLVER'}

\medskip

\noindent Check 'inparam\_basic' (set the value for MESHNAME to the meshname from above, among others), and also (differences with *.TEMPLATES provided
by AxiSEM) :

\begin{itemize}

\item[\textbullet] ATTENUATION         false
\item[\textbullet] SAVE\_SNAPSHOTS     true

\end{itemize}

\medskip

\noindent Check 'inparam\_advanced' :

\begin{itemize}

\item[\textbullet] KERNEL\_WAVEFIELDS   true
\item[\textbullet] KERNEL\_IBEG         0
\item[\textbullet] KERNEL\_IEND         4

\end{itemize}

\medskip

\noindent Type : ./sub\_launch\_script\_for\_run\_AxiSEM\_Curie\_CD.csh NAME\_OF\_RESULTS\_DIR

\smallskip

\noindent !! CAUTION : don't forget "NAME\_OF\_RESULTS\_DIR" (it will be your results folder, and a subfolder of SOLVER/) !!

\smallskip

\noindent It run, on N procs in batch on Curie, the compilation and the execution of  AxiSEM solver.
Among others, this script calls : \\

\noindent ./add\_line\_number\_in\_points\_lists.sh (rename files 'list\_gll\_boundary\_spherical.txt' \& \\ 'list\_gll\_boundary\_cartesian.txt', and add number of lines) \\

\noindent run ./sub\_called\_batch\_for\_AxiSEM\_Curie\_CD.sh (submit the job on Curie)


\section{STEP 4 : Specfem Partitionning ~ \& ~ STEP 5 : Specfem Generate database}

\noindent Go back in 'EXAMPLES/coupling\_with\_EXTERNAL\_CODES/with\_AxiSEM/example\_1st\_for\_validation'

\medskip

\noindent Type : run ./batch\_coupling\_step4\_and\_step5\_scotch\_part\_and\_generdata\_for\_AxiSEM.sh

\medskip

\noindent !! CAUTION : the number of procs in DATA/Par\_file, have to be the same as in the 'inparam\_mesh' (cf STEP 2) of AxiSEM !!


\section{Interface $\Longrightarrow$ STEP 6 : expand 2D to 3D ~ \& ~ STEP 7 : reformat}

\noindent Go in {\scriptsize 'utils/EXTERNAL\_CODES\_coupled\_with\_SPECFEM3D/AxiSEM\_for\_SPECFEM3D/AxiSEM\_modif\_for\_coupling\_with\_specfem/SOLVER'}

\medskip

\begin{itemize}

\item[\textbullet] {\color{red} Check the file 'expand\_2D\_3D.par', cf THE PATHS IN THE 3 LAST LINES (normally, these are the only paths you to modify in the AxiSEM dir) !!} ==> Caution : for the tractions path (Tractions/1/), the directory need to be created !! ==>

it is created in the script 'sub\_launch\_script\_for\_make\_+\_run\_expand2D3D\_and\_reformat\_Curie\_CD.sh', but if you don't use this script, you have to create the directory yourself

\smallskip

\noindent Example :
\begin{verbatim}

input_box.txt
input_box_sem_cart.txt
32                          # number of AxiSEM mpi processes used in solver
90. 0.                      # source position (lat lon)
0.  60.                     # chunk center (lat lon)
1                           # number of axisem simus depends on moment tensor used
32                          # number of Specfem3D MPI processes
/[...]/example_1st_for_validation/MESH                          # Specfem MESH dir
/[...]/example_1st_for_validation/OUTPUT_FILES/DATABASES_MPI    # Specfem databases dir
/[...]/example_1st_for_validation/DATA/AxiSEM_tractions/1       # AxiSEM tractions dir

\end{verbatim}

\item[\textbullet] Check the file 'reformat.par'

\smallskip

\noindent Example :
\begin{verbatim}

10.         # = 1/DT, i.e. output sampling in Hz (time step that will use in Specfem3D simu)
710. 1800.  # begin time and end time (s.)

\end{verbatim}

\end{itemize}

\medskip

\noindent You can launch the compilation (of expand\_2D\_3D and reformat) AND the execution of this two programs with the single script :

\smallskip

\noindent ./sub\_launch\_script\_for\_make\_+\_run\_expand2D3D\_and\_reformat\_Curie\_CD.sh NAME\_OF\_RESULTS\_DIR

\medskip

\noindent !! CAUTION : don't forget "NAME\_OF\_RESULTS\_DIR" !!

\medskip

\noindent It will do the compilation, launch ./sub\_called\_batch\_for\_expand2D3D\_Curie\_CD.sh, and at the end, launch ./sub\_called\_batch\_for\_reformat\_Curie\_CD.sh $\Longleftarrow$ IMPORTANT : for the expand\_2D\_3D part, the number processes is arbitrary, BUT for the reformat part you have to configure the batch with the **SAME** number of processes that Specfem3D will use. Onefile is created by one process for one Spefem3D partion of domain.

\smallskip

\noindent All the results will be in 'NAME\_OF\_RESULTS\_DIR'.

\bigskip

\noindent If you prefer you do this step by step :

\begin{itemize}

\item[\textbullet] Go in {\scriptsize 'utils/EXTERNAL\_CODES\_coupled\_with\_SPECFEM3D/AxiSEM\_for\_SPECFEM3D/UTILS\_COUPLING\_SpecFEM'} to make the compilation.
\item[\textbullet] Go in NAME\_OF\_RESULTS\_DIR and type : run ../sub\_called\_batch\_for\_expand2D3D\_Curie\_CD.sh to submit expand\_2D\_3D
\item[\textbullet] Go in NAME\_OF\_RESULTS\_DIR and type : run ../sub\_called\_batch\_for\_reformat\_Curie\_CD.sh to submit reformat

\end{itemize}

\section{STEP 8 : Running specfem3D}

\noindent Go in 'EXAMPLES/coupling\_with\_EXTERNAL\_CODES/with\_AxiSEM/example\_1st\_for\_validation'

\medskip

\noindent Check in 'DATA/Par\_file', the lines after the comment "\# time step parameters", these parameters must coincide with the parameters in  'reformat.par' (cf STEP 7) :

\begin{itemize}

\item[\textbullet] NSTEP = 10900 ~~ \# $\Longrightarrow$ cf the two values in the second line of 'reformat.par'. This value equals to :
$$\ds \dfrac{\hbox{second value} - \hbox{the first}}{\text{DT}} ~~ \left(= \dfrac{\hbox{duration of physical time}}{\text{DT}} \right)$$

\noindent So here, it is : $\ds \dfrac{1800 - 710}{0.1} = 10900$

\medskip

\item[\textbullet] DT = 0.1 ~~ \# $\Longrightarrow$ the inverse of the first line in 'reformat.par' (which is 1/DT)

\end{itemize}

\medskip

\noindent Also check in 'DATA/Par\_file', the lines after the comment "\# to couple with an external code (such as DSM, AxiSEM, or FK)" :

\begin{itemize}

\item[\textbullet] COUPLE\_WITH\_EXTERNAL\_CODE = .true. ~~ \# $\Longrightarrow$ always set it to '.true.' in this case

\item[\textbullet] EXTERNAL\_CODE\_TYPE = 2  ~ \# 1 = DSM, 2 = AxiSEM, 3 = FK ~~ $\Longrightarrow$ always set it to '2' for AxiSEM

\item[\textbullet] TRACTION\_PATH  = ./DATA/AxiSEM\_tractions/1/ ~~   \# $\Longrightarrow$ This is where there are the tractions from AxiSEM (from steps 6 and 7) !! CAUTION : verify that they are in this dir, in the form of files 'proc0000**\_sol\_axisem', etc. !!

\item[\textbullet] MESH\_A\_CHUNK\_OF\_THE\_EARTH  = .true.  ~~  \# $\Longrightarrow$ always set it to '.true.' in this case

\end{itemize}

\medskip

\noindent And finally, type : run ./batch\_coupling\_step8\_xspecfem3d.sh

\section{STEP ? : Set up scripts}


\end{document}
