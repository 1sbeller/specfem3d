\subsection{PyAxi: Python interface for AXISEM}

\subsection{Introduction}
PyAxi is a Python script developed as an interface for AXISEM. 
All the options available in AXISEM are included in only one input file (\textit{inpython.cfg}).
By running the script, all the necessary steps (MESHER, SOLVER and Post-Processing) will be done automatically.
Python is the only requirement; However, some special functionalities (mseed format, plotting in Python environment) need \textit{ObsPy} to be installed. \\

\noindent \textit{Basic requirement}: Python.\\
\textit{Convert to MSEED (not obligatory)}: ObsPy (https://github.com/obspy/obspy/wiki).\\

\subsubsection{How to Run PyAxi?}
%In this part, we show how to run PyAxi for different input files.
First, let's check whether AXISEM can be run properly on our machine. 
For this reason:\\

Start from within the {\tt AXISEM} directory:
\begin{enumerate}
\itemsep0em
\item {\tt cd TESTING}
\item {\tt python PyAxi.py --check}
\end{enumerate}
\noindent This gives an overview of the installation status of all relevant compilers and tools required
to run AXISEM on your machine (Figure~\ref{check_pyaxi}). 
Please note that \textit{--check} does not check for all the possible compilers.
It just checks for those listed in Figure~\ref{check_pyaxi}; 
therefore, for other compilers, it should be done manually.\\

\begin{figure*}[htb]
\begin{center}
\begin{verbatim}
 
   -----------------------------------------
		   PyAxi
	   Python interface for Axisem
   -----------------------------------------

Check the Basic, Processing and Visualization requirements:

          Compiler    |    Installed
  ------------------------------------------
          gfortran    |        Y
             ifort    |        Y
    mpif90.openmpi    |        Y
            mpif90    |        Y
           gnuplot    |        Y
              taup    |        Y
          paraview    |        Y
            matlab    |        Y
       googleearth    |        Y

Summary:

Basic functionality requirement...CHECK
Processing requirement........CHECK
Visualization tools...CHECK
\end{verbatim}
\end{center}
\caption{\textit{Checking all the relevant compilers and tools required to run AXISEM.}}
\label{check_pyaxi}
\end{figure*}


\noindent After installation of all required packages, 
maybe the best way to get familiar with AXISEM is to run the code with the default input file:\\

Start from within the {\tt AXISEM} directory:
\begin{enumerate}
\itemsep0em
\item {\tt cd TESTING}
\item {\tt python PyAxi.py inpython.cfg}
\end{enumerate}

All the rest will be done automatically...\\

\noindent \textit{inpython.cfg} is a configuration file that contains all the AXISEM options.
To change the input file, open \textit{inpython.cfg} with an editor:\\

Start from within the {\tt AXISEM} directory:
\begin{enumerate}
\itemsep0em
\item {\tt cd TESTING}
\item {\tt (editor) inpython.cfg}
\end{enumerate}


%%%and it has been divided into several parts:
%%%\subsubsection{general}
%%%This section in \textit{inpython.cfg} is dedicated to how and where you want to run the code.
%%%\begin{itemize}
%%% \item \textbf{address} is the directory where you have the AXISEM code.
%%% \item \textbf{mesh\_name} is the name of the directory in which the info of the generated mesh will be stored.
%%% \item \textbf{solver\_name} is the name of the directory in which the final solution will be stored.
%%% \item \textbf{verbose} produces verbose output on the screen (recommended for debugging).
%%% \item \textbf{new\_mesh} has three possibilities.
%%%'Y' runs all steps of AXISEM from generating the mesh up to saving the waveforms.
%%%'N' uses the available mesh and continues the code.
%%%'M' gives this possibility to the user to manually change the options listed at the end of the \textbf{general} section.
%%% \item \textbf{post\_processing} perferms the post processing step automatically.
%%%\end{itemize}
%%%
%%%\noindent Please note that these options make it possible for the user to change the work-flow as it is required.
%%%For instance, if you already have done one simulation and you want to use the same mesh for another simulation,
%%%it is enough to set $new\_mesh = N$.
%%%
%%%\subsubsection{mpi\_netCDF}
%%%mpi\_netCDF options control the make flags and netCDF functionality.
%%%\begin{itemize}
%%% \item \textbf{make\_flag} adds required flag(s) for running the \textit{makemake.pl} in both MESHER and SOLVER.
%%% \item \textbf{mpi\_compiler} could be set based on your local machine. 
%%% \item \textbf{netCDF} generates one netCDF file instead of having binary output.
%%% \item \textbf{netCDF\_LIBS} and \textbf{netCDF\_INCLUDE} should be changed according to your netCDF installation.
%%%\end{itemize}
%%%\subsubsection{mesher}
%%%Major options that control the MESHER part of AXISEM have been included in this part.
%%%For more information about \textbf{model}, \textbf{period} and \textbf{no\_proc} please refer to Mesher input section.
%%%
%%%\subsubsection{solver}
%%%In this part, first we have three options \textbf{no\_simu}, \textbf{seis\_length} and \textbf{time\_step}
%%%that are identical to \textbf{number of simulations}, \textbf{seismogram length} and 
%%%\textbf{time step} defined in Solver input. Moreover:
%%%\begin{itemize}
%%% \item \textbf{source\_type} could be selected from 'sourceparams' and 'cmtsolut'.
%%% \item \textbf{receiver\_type} has three 'colatlan', 'stations' and 'database' options.
%%% \item \textbf{save\_XDMF} saves XDMF files (high resolution 2D wavefields), more options in \textit{inparam\_xdmf}.
%%% \item \textbf{force\_aniso} for anisotropic model handling.
%%%\end{itemize}
%%%\noindent Based on what we have selected in 'source\_type', one of the two parts for the source parameters
%%%should be modified, e.g. if you have chosed 'cmtsolut', then go to the \textit{cmtsolut} parameters and 
%%%change the options accordingly.
%%%\subsubsection{post\_processing}
%%%This section controls the required options for Post processing. All the parameters are identical to what has been explained in \textit{Post processing} section.
%%%\subsubsection{MISC}
%%%MISC contains the input parameters for converting the waveforms to MSEED format and convolve them with Source Time Function.
%%%These parameters are optional and need \textit{ObsPy} to be installed.
%%%\begin{itemize}
%%% \item \textbf{mseed} to convert all the seismograms to MSEED format (one file for each).
%%%These files will be located in the SEISMOGRAMS/MSEED folder in each solution directory (\textit{solver\_name}).
%%% \item \textbf{mseed\_all} to convert all the seismograms into MSEED format (one file for all).
%%%It will generate one 'seismograms.mseed' file saved in SEISMOGRAMS folder. 
%%% \item \textbf{convSTF} convolves the converted seismograms with Source Time Function (STF).
%%% \item \textbf{halfduration} determines the halfduration of the STF.
%%% \item \textbf{filter} applies a lowpass and a highpass filter with the minimum and maximum
%%%frequencies defined in \textbf{fmin} and \textbf{fmax}.
%%%\end{itemize}
%%%\subsubsection{test}
%%%This part of the input file is just for the TESTING functionality that we have in AXISEM.
%%%In normal runs, one should keep the \textit{'test'} flag to 'N' to avoid any problem.
%%%TESTING will be discussed in a seperated section.

