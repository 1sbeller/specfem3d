\chapter[Doing tomography based on the sensitivity kernels obtained]{Doing tomography, i.e., updating the model based on the sensitivity kernels obtained}\label{cha:tomo}

UNDER CONSTRUCTION (April 2015). Will be added around June 2015.

\section{Tomographic full waveform inversion (FWI) / imaging using the sensitivity kernels obtained}

One of the fundamental reasons for computing sensitivity kernels (Section~\ref{sec:Adjoint-simulation-finite})
is to use them within a tomographic inversion, i.e., perform imaging. In other words, use
recorded seismograms, make measurements with synthetic seismograms,
and use the misfit between the two to iteratively improve the model
described by (at least) $V_{{\rm p}}$, $V_{{\rm s}}$, and $\rho$ as a function of space.

As explained for instance in \cite{MoChKoWa15},
full waveform inversion means that one considers the observed seismograms (possibly filtered) as the
basic observables that one wants to fit. One thus searches for the model that minimizes the mean squared difference between observed
and synthetic seismograms. In other words, the goal is to find a structural model that can explain a larger portion
of seismological records, and not simply the phase of a few seismic arrivals.

Whatever misfit function you use for the tomographic inversion (you can find several
examples in \citet{TrKoLi08} and \citet{TrTaLi05}), you will weight
the sensitivity kernels with measurements. Furthermore, you will use
as many measurements (stations, components, time windows) as possible
per event; hence, we call these composite kernels ``event kernels'',
which are volumetric fields representing the gradient of the misfit
function with respect to one of the variables (\eg $V_{{\rm s}}$).
The basic features of an adjoint-based tomographic inversion were
illustrated in \citet{TrKoLi08} and \citet{TaLiTr07} using a conjugate-gradient algorithm.
Other, more powerful techniques such as the Limited-Memory Broyden-Fletcher-Goldfarb-Shanno (L-BFGS) method can now be used,
as illustrated for instance in \cite{MoChKoWa15}.

\subsection{Principle}

We want to minimize the classical waveform misfit function:
\begin{equation}
\chi \left( \boldm \right) = \sum_{s=1}^N\sum_{r=1}^M \int_0^T \frac{1}{2} \|\boldu(\boldx_r, \boldx_s; t) - \boldd(\boldx_r, \boldx_s; t)\| ^{2} \, dt .
\label{misfit}
\end{equation}
This functional quantifies the $L^2$ difference between the observed waveforms $\boldd(\boldx_r, \boldx_s; t)$ at receivers $\boldx_r$, $r = 1, ..., M$
produced by sources at $\boldx_s$, $s=1,...,N$, and the corresponding synthetic seismograms $\boldu(\boldx_r, \boldx_s; t)$ computed in model $\boldm$.
While this misfit function is indeed classical, it is worth mentioning that in the case of noisy real data other norms could be used,
since in the oil industry for instance it is known that the $L^1$ norm
(\cite{CrPiNoMcTa90,Brossier_2010_WDR}), hybrid $L^1-L^2$ norms (\cite{Bube_1997_HL1}), Hubert norm (\cite{Ha_2009_WIU}),
Student-t distribution (\cite{ArVaHe11,Jeong_2015}) etc... can be more robust that the $L^2$ norm used here in the context of synthetic data with no noise.
In the vicinity of $\boldm$, the misfit function can be expanded into a Taylor series:
\begin{equation}
\chi(\boldm+ \delta\boldm) \approx \chi(\boldm)+\boldg(\boldm)\cdot\delta\boldm
+\delta\boldm\cdot\mathbf{H}(\boldm)\cdot\delta\boldm \, ,
\label{local_misfit}
\end{equation}
where $\boldg(\boldm)$ is the gradient of the waveform misfit function:
\begin{equation}
\boldg(\boldm) = \frac{\partial\chi(\boldm)}{\partial\boldm} \, ,
\end{equation}
and $\mathbf{H}(\boldm)$ the Hessian:
\begin{equation}
\mathbf{H}(\boldm) = \frac{\partial^2\chi(\boldm)}{\partial\boldm^2} \, .
\end{equation}
In the following, for simplicity the dependence of the gradient and Hessian on the model will be implicitly assumed and omitted in the notations.
The nearest minimum of $\chi$ in (\ref{local_misfit}) with respect to the model perturbation $\delta \boldm$ is reached for
\begin{equation}
\delta \boldm = -\mathbf{H}^{-1} \cdot \boldg \, .
\label{Newton}
\end{equation}
The local minimum of (\ref{misfit}) is thus given by perturbing the model in the direction of the gradient preconditioned
by the inverse Hessian.

\subsection{Computation of the gradient based on the adjoint method}

A direct method to compute the gradient is to take the derivative of (\ref{misfit}) with respect to model parameters:
\begin{equation}
\frac{\partial\chi(\boldm)}{\partial\boldm} = -\sum_{s=1}^N\sum_{r=1}^M \int_0^T
\frac{\partial \boldu(\boldx_r, \boldx_s; t)}{\partial\boldm}\cdot\left[\boldu(\boldx_r, \boldx_s; t) - \boldd(\boldx_r, \boldx_s; t)\right] \, dt \, .
\label{calc_gradient}
\end{equation}
This equation can be reformulated as the matrix-vector product:
\begin{equation}
\boldg = -\mathbf{J}^* \cdot \delta \mathbf{d} \, ,
\label{Jacobian}
\end{equation}
where $\mathbf{J}^*$ is the adjoint of the Jacobian matrix of the forward problem that contains the Fr\'echet derivatives of the
data with respect to model parameters, and $\delta \mathbf{d}$ is the vector that contains the data residuals.
The determination of $\mathbf{J}$ would require computing the Fr\'echet derivatives for each time step in the time
window considered and for all the source-station pairs, which is completely prohibitive on current supercomputers
(let us note that this situation may change one day).
However, it is possible to obtain this gradient without computing the Jacobian matrix explicitly.
The approach to determine the gradient without computing the Fr\'echet derivatives was introduced in nonlinear optimization
by \cite{Cha74} working with J. L. Lions, and later applied to seismic exploration problems by
\cite{BaChLa77}, \cite{BaChHeLa82}, \cite{Lai83} and \cite{Tar84}. The idea is to resort to the adjoint state,
which corresponds to the wavefield emitted and back-propagated from the receivers \cite[e.g.,][]{TrTaLi05,TrKoLi08,Plessix_2006_RAS,FiBuIg06}.

Let us give an outline of the theory to compute the gradient based on the adjoint method, and refer the reader to e.g. \cite{TrTaLi05}
and \cite{TrKoLi08} for further details. The perturbation of the misfit function can be expressed as:
\begin{equation}
\delta \chi (\boldm) = \sum_{s=1}^N\sum_{r=1}^M \int_0^T \left[\boldu(\boldx_r, \boldx_s; t) - \boldd(\boldx_r, \boldx_s; t)\right]\cdot \delta
\boldu(\boldx_r, \boldx_s; t) \, dt \, ,
\label{dchi}
\end{equation}
where $\delta\boldu$ is the perturbation of displacement given by the first-order Born approximation \cite[e.g.,][]{Hud77}:
\begin{eqnarray}
\delta \boldu (\boldx_r, \boldx_s; t) & = & -\int_0^t \int_V
\left[\delta\rho(\boldx)\mathbf{G}(\boldx_r,\boldx;t-t')\cdot\partial^2_{t'}\boldu(\boldx,\boldx_s;t') \right. \nonumber \\
& + & \left. \nabla \mathbf{G}(\boldx_r,\boldx;t-t'):\delta \boldc(\boldx): \nabla\boldu(\boldx;t') \right] \, d^3\boldx \, dt' \, .
\label{Born}
\end{eqnarray}
In this expression, $\mathbf{G}$ is the Green's tensor, $\delta\rho$ the perturbation of density, $\delta c$ the perturbation of the fourth-order elasticity
tensor,
and a colon denotes a double tensor contraction operation.
Inserting (\ref{Born}) into (\ref{dchi}) we obtain
\begin{eqnarray}
\delta \chi (\boldm) & = & -\sum_{s=1}^N\sum_{r=1}^M \int_0^T \left[\boldu(\boldx_r, \boldx_s; t) - \boldd(\boldx_r, \boldx_s; t)\right] \int_0^t \int_V
\left[\delta\rho(\boldx)\mathbf{G}(\boldx_r,\boldx;t-t')\cdot\partial^2_{t'}\boldu(\boldx,\boldx_s;t') \right. \nonumber \\
& + & \left. \nabla \mathbf{G}(\boldx_r,\boldx;t-t'):\delta \boldc(\boldx):\nabla\boldu(\boldx,t') \right] \, d^3\boldx \, dt' \, dt \,.
\end{eqnarray}
Defining the waveform adjoint source for each source $\boldx_s$
\begin{equation}
\boldf^\dagger(\boldx,\boldx_s;t) = \sum_{r=1}^M  \left[\boldu(\boldx_r, \boldx_s; T-t) - \boldd(\boldx_r, \boldx_s; T-t)\right] \delta(\boldx-\boldx_r) \, ,
\end{equation}
and the corresponding adjoint wavefield
\begin{equation}
\boldu^\dagger(\boldx,\boldx_s;t) = \int_0^{t'}\int_V\mathbf{G}(\boldx,\boldx';t'-t)\cdot\boldf^\dagger(\boldx',\boldx_s;t)\, d^3\boldx' \, dt \, ,
\end{equation}
the perturbation of the misfit function may be expressed as:
\begin{eqnarray}
\delta\chi(\boldm) & = & - \sum_{s=1}^N\nonumber \int_V \int_0^T \left[\delta\rho \, \boldu^\dagger(\boldx,\boldx_s;T-t)\cdot
\partial^2_{t}\boldu(\boldx,\boldx_s;t) \right. \nonumber \\
& +  & \left. \nabla\boldu^\dagger(\boldx,\boldx_s;T-t):\delta \boldc: \nabla \boldu(\boldx,\boldx_s;t) \right] \, d^3\boldx \, dt \, .
\label{dchi2}
\end{eqnarray}
At this point, we make some assumptions on the nature of the elasticity tensor.
A general fourth-order elasticity tensor is described by 21 elastic parameters, a very large number that makes its complete characterization
way beyond the reach of any tomographic approach. For the time being, let us thus consider isotropic elasticity tensors,
described by the two Lam\'e parameters $\lambda$ and $\mu$:
\begin{equation}
c_{ijkl} = \lambda \delta_{ij}\delta_{kl}+\mu(\delta_{ik}\delta_{jl}+\delta_{il}\delta_{jk}) \, .
\end{equation}
In this case, (\ref{dchi2}) can be written as:
\begin{eqnarray}
\delta\chi(\boldm) & = & - \sum_{s=1}^N\nonumber \int_V \left[ K_\rho(\boldx,\boldx_s)\delta\ln\rho(\boldx)
+K_\lambda(\boldx,\boldx_s)\delta\ln\lambda(\boldx) \right. \nonumber \\
& + & \left. K_\mu(\boldx,\boldx_s)\delta\ln\mu(\boldx)\right] \, d^3\boldx \, ,
\label{dchi2_1}
\end{eqnarray}
where ln() is the natural logarithm and where the Fr\'echet derivatives with respect to the density and Lam\'e parameters are given by:
\begin{eqnarray}
K_\rho(\boldx,\boldx_s) & = & - \int_0^T \rho(\boldx) \boldu^\dagger(\boldx,\boldx_s;T-t)\cdot \partial^2_{t}\boldu(\boldx,\boldx_s;t) \, dt \\
K_\lambda(\boldx,\boldx_s) & = & - \int_0^T \lambda(\boldx) \nabla\cdot\boldu^\dagger(\boldx,\boldx_s;T-t) \nabla\cdot\boldu(\boldx,\boldx_s;t) \, dt \\
K_\mu(\boldx,\boldx_s) & = & -2 \int_0^T \mu(\boldx) (\boldx) \nabla\boldu^\dagger(\boldx,\boldx_s;T-t) :\nabla\boldu(\boldx,\boldx_s;t)\, dt \, .
\end{eqnarray}
Since the propagation of seismic waves mainly depends on compressional wave speed $\alpha$ and shear wave speed $\beta$,
but also because these seismic velocities are easier to interpret, tomographic models are usually described based on these
two parameters. With this new parametrization, the perturbation of the misfit function may be written as:
\begin{eqnarray}
\delta\chi(\boldm) & = & - \sum_{s=1}^N\nonumber \int_V \left[ K'_\rho(\boldx,\boldx_s)\delta\ln\rho(\boldx)
+K'_\alpha(\boldx,\boldx_s)\delta\ln\alpha(\boldx) \right. \nonumber \\
& + & \left. K'_\beta(\boldx,\boldx_s)\delta\ln\beta(\boldx)\right] \, d^3\boldx \, ,
\end{eqnarray}
where
\begin{eqnarray}
K'_\rho(\boldx,\boldx_s) & = &  K_\rho(\boldx,\boldx_s)+K_\lambda(\boldx,\boldx_s)+K_\mu(\boldx,\boldx_s) \\
K'_\alpha(\boldx,\boldx_s) & = &  2 \left(\frac{\lambda+2\mu}{\lambda}\right) K_\lambda(\boldx,\boldx_s) \\
K'_\beta(\boldx,\boldx_s) & = & 2 K_\mu -\frac{4\mu}{\lambda} K_\lambda (\boldx,\boldx_s) \, .
\end{eqnarray}

As can be seen from these expressions, the principle of the adjoint-state method is to correlate two wavefields:
the direct (i.e. forward) field that propagates from the source to the receivers, and the adjoint field that propagates from all the
receivers backward in time. The same approach can be followed for any type of seismic observable (phase, amplitude,
envelope, time series...), provided the appropriate adjoint source is used \cite[]{TrTaLi05,TrKoLi08}.
For example, for the cross-correlated traveltime of a seismic
phase, the adjoint source is defined as the velocity of that synthetic phase weighted by the travel-time residual.

Computing the gradient based on the adjoint-state method requires performing two simulations per source (forward and adjoint
fields) regardless of the type of observable. However, to define the adjoint field one
must know the adjoint source, and that source is computed from the results of the forward simulation. One must therefore perform
the forward simulation before the adjoint simulation.
A straightforward solution for time-domain methods would be to store the whole forward field to disk at
each time step during the forward run and then read it back during the adjoint simulation to calculate the interaction of these two fields.
In 2-D this is feasible but in the 3-D case for very short seismic periods and without lossy compression, downsampling,
or a large amount of disk or memory checkpointing \cite[e.g.,][]{FiKeIg09,RuHaPuGu14,CyShWi15}
the amount of disk storage required would currently be too large.
However let us note again that this situation will change in the future.
In the mean time, a standard possible solution is to perform three simulations per source \cite[]{TrKoLi08,PeKoLuMaLeCaLeMaLiBlNiBaTr11},
i.e., perform the forward calculation twice, once to compute the adjoint sources and once again
at the same time as the adjoint simulation to correlate the two fields and sum their interaction on the fly over all the time steps.
Doing so for an elastic Earth, one only needs a small amount of disk storage to store the last time
step of the forward run, which is then used as an initial condition to redo the forward run backwards,
as well as the field on the outer edges of the mesh for each time steps in order to be able to undo the absorbing boundary conditions.

\section{OLD VERSION OF THE SECTION, WILL SOON BE IMPROVED AND ENRICHED}

There are dozens of versions of gradient-based inversion
algorithms that could alternatively be used. The tomographic inversion
of \citet{TaLiMaTr09,TaLiMaTr2010} used SPECFEM3D Cartesian as well
as several additional components which are also stored on the CIG
svn server, described next.

The directory containing external utilities for tomographic inversion using
SPECFEM3D Cartesian (or other packages that evaluate misfit functions
and gradients) is in directory \texttt{utils/ADJOINT\_TOMOGRAPHY\_TOOLS/}:
\begin{verbatim}
flexwin/     -- FLEXWIN algorithm for automated picking of time windows
measure_adj/ -- reads FLEXWIN output file and makes measurements,
                with the option for computing adjoint sources
iterate_adj/ -- various tools for iterative inversion
                (requires pre-computed "event kernels")
\end{verbatim}
This directory also contains
a brief \verb+README+ file indicating the role of the three subdirectories,
\verb+flexwin+ \citep{Maggi2009}, \verb+measure_adj+, and \verb+iterate_adj+.
The components for making the model update are there; however, there
are no explicit rules for computing the model update, just as with
any optimization problem. There are options for computing a conjugate
gradient step, as well as a source subspace projection step.

The best single file to read is probably: \verb+ADJOINT_TOMO/iterate_adj/cluster/README+.


