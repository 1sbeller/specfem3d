\chapter{Introduction}

The software package SPECFEM3D Cartesian simulates seismic wave propagation
at the local or regional scale and performs full waveform imaging (FWI) or adjoint tomography based upon the spectral-element method
(SEM). The SEM is a continuous Galerkin technique \citep{TrKoLi08,PeKoLuMaLeCaLeMaLiBlNiBaTr11},
which can easily be made discontinuous \citep{BeMaPa94,Ch00,KoWoHu02,ChCaVi03,LaWaBe05,Kop06,WiStBuGh10,AcKo11};
it is then close to a particular case of the discontinuous Galerkin
technique \citep{ReHi73,LeRa74,Arn82,JoPi86,BoMaHe91,FaRi99,HuHuRa99,CoKaSh00,GiHeWa02,RiWh03,MoRi05,GrScSc06,AiMoMu06,BeLaPi06,DuKa06,DeSeWh08,PuAmKa09,WiStBuGh10,DeSe10,EtChViGl10},
with optimized efficiency because of its tensorized basis functions
\citep{WiStBuGh10,AcKo11}. In particular, it can accurately handle
very distorted mesh elements \citep{OlSe11}.\\

\red{In fluids, when gravity is turned off, SPECFEM3D uses the classical
linearized Euler equation; thus if you have sharp local variations of
density in the fluid (highly heterogeneous fluids
in terms of density) or if density becomes extremely small in some
regions of your model (e.g. for upper-atmosphere studies), before using
the code please make sure the linearized Euler equation is a valid
approximation in the case you want to study, and/or see if you should turn gravity on.
For more details on that see e.g. \cite{COA2011}.}\\

It has very good accuracy and convergence properties \citep{MaPa89,SePr94,DeFiMu02,Coh02,DeSe07,SeOl08,AiWa09,AiWa10,MeStTh12}.
The spectral element approach admits spectral rates of convergence
and allows exploiting $hp$-convergence schemes. It is also very well
suited to parallel implementation on very large supercomputers \citep{KoTsChTr03,TsKoChTr03,KoLaMi08a,CaKoLaTiMiLeSnTr08,KoViCh10}
as well as on clusters of GPU accelerating graphics cards \citep{Kom11,MiKo10,KoMiEr09,KoErGoMi10}.
Tensor products inside each element can be optimized to reach very
high efficiency \citep{DeFiMu02}, and mesh point and element numbering
can be optimized to reduce processor cache misses and improve cache
reuse \citep{KoLaMi08a}. The SEM can also handle triangular (in 2D)
or tetrahedral (in 3D) elements \citep{WinBoyd96,TaWi00,KoMaTrTaWi01,Coh02,MeViSa06}
as well as mixed meshes, although with increased cost and reduced
accuracy in these elements, as in the discontinuous Galerkin method.\\


Note that in many geological models in the context of seismic wave
propagation studies (except for instance for fault dynamic rupture
studies, in which very high frequencies or supershear rupture need
to be modeled near the fault, see e.g. \citet{BeGlCrViPi07,BeGlCrVi09,PuAmKa09,TaCrEtViBeSa10})
a continuous formulation is sufficient because material property contrasts
are not drastic and thus conforming mesh doubling bricks can efficiently
handle mesh size variations \citep{KoTr02a,KoLiTrSuStSh04,LeChLiKoHuTr08,LeChKoHuTr09,LeKoHuTr09}.\\


For a detailed introduction to the SEM as applied to regional seismic
wave propagation, please consult \citet{PeKoLuMaLeCaLeMaLiBlNiBaTr11,TrKoLi08,KoVi98,KoTr99,ChKoViCaVaFe07}
and in particular \citet{LeKoHuTr09,LeChKoHuTr09,LeChLiKoHuTr08,GoAmTaCaSmSaMaKo09,WiKoScTr04,KoLiTrSuStSh04}.
A detailed theoretical analysis of the dispersion
and stability properties of the SEM is available in \citet{Coh02}, \citet{DeSe07}, \citet{SeOl07}, \citet{SeOl08} and \citet{MeStTh12}.\\


Effects due to lateral variations in compressional-wave speed, shear-wave
speed, density, a 3D crustal model, topography and bathymetry are
included. The package can accommodate full 21-parameter anisotropy
(see~\citet{ChTr07}) as well as lateral variations in attenuation
\citep{SaKoTr10}. Adjoint capabilities and finite-frequency kernel
simulations are included \citep{TrKoLi08,PeKoLuMaLeCaLeMaLiBlNiBaTr11,LiTr06,FiIgBuKe09,ViOp09}.\\


The SEM was originally developed in computational fluid dynamics \citep{Pat84,MaPa89}
and has been successfully adapted to address problems in seismic wave
propagation. Early seismic wave propagation applications of the SEM,
utilizing Legendre basis functions and a perfectly diagonal mass matrix,
include \citet{CoJoTo93}, \citet{Kom97}, \citet{FaMaPaQu97}, \citet{CaGa97},
\citet{KoVi98} and \citet{KoTr99}, whereas applications involving
Chebyshev basis functions and a non-diagonal mass matrix include \citet{SePr94},
\citet{PrCaSe94} and \citet{SePrPr95}.\\


All SPECFEM3D software is written in Fortran2003 with full portability
in mind, and conforms strictly to the Fortran2003 standard. It uses
no obsolete or obsolescent features of Fortran. The package uses parallel
programming based upon the Message Passing Interface (MPI) \citep{GrLuSk94,Pac97}.\\


SPECFEM3D won the Gordon Bell award for best performance at the SuperComputing~2003
conference in Phoenix, Arizona (USA) (see \citet{KoTsChTr03} and
\url{www.sc-conference.org/sc2003/nrfinalaward.html}). It was a finalist
again in 2008 for a run at 0.16 petaflops (sustained) on 149,784 processors
of the `Jaguar' Cray XT5 system at Oak Ridge National Laboratories
(USA) \citep{CaKoLaTiMiLeSnTr08}. It also won the BULL Joseph Fourier
supercomputing award in 2010.\\

It reached the sustained one petaflop performance level for the first time in February 2013
on the Blue Waters Cray supercomputer at the National Center for Supercomputing Applications (NCSA), located at the University of Illinois at Urbana-Champaign (USA).\\


This new release of the code includes Convolution or
Auxiliary Differential Equation Perfectly Matched absorbing Layers
(C-PML or ADE-PML) \citep{MaKoEz08,MaKoGe08,MaKo09,MaKoGeBr10,KoMa07}.
It also includes support for GPU graphics card acceleration \citep{Kom11,MiKo10,KoMiEr09,KoErGoMi10}.\\


The next release of the code will use the PT-SCOTCH parallel and threaded
version of SCOTCH for mesh partitioning instead of the serial version.\\


SPECFEM3D Cartesian includes coupled fluid-solid domains and adjoint
capabilities, which enables one to address seismological inverse problems,
but for linear rheologies only so far. To accommodate visco-plastic
or non-linear rheologies, the reader can refer to the \href{http://geoelse.stru.polimi.it/}{GeoELSE}software
package \citep{CaGa97,StPaIg09}.


\section{Citation}

If you use SPECFEM3D Cartesian for your own research, please cite
at least one of the following articles:
\begin{description}
\item [{Numerical simulations in general}] ~\\
 Forward and adjoint simulations are described in detail in \citet{TrKoLi08,PeKoLuMaLeCaLeMaLiBlNiBaTr11,VaCaSaKoVi99,KoMiEr09,KoErGoMi10,ChKoViCaVaFe07,MaKoDi09,KoViCh10,CaKoLaTiMiLeSnTr08,TrKoHjLiZhPeBoMcFrTrHu10,KoRiTr02,KoTr02a,KoTr02b,KoTr99}
or \citet{KoVi98}. Additional aspects of adjoint simulations are
described in \citet{TrTaLi05,LiTr06,TrKoLi08,LiTr08,TrKoHjLiZhPeBoMcFrTrHu10,PeKoLuMaLeCaLeMaLiBlNiBaTr11}.
Domain decomposition is explained in detail in \citet{MaKoBlLe08},
and excellent scaling up to 150,000 processor cores is shown for instance
in \citet{CaKoLaTiMiLeSnTr08,KoLaMi08a,MaKoBlLe08,KoErGoMi10,Kom11},
\item [{GPU computing}] ~\\
 Computing on GPU graphics cards for acoustic or seismic wave propagation
applications is described in detail in \citet{Kom11,MiKo10,KoMiEr09,KoErGoMi10}.
\end{description}
\noindent If you use this new version, which has non blocking
MPI for much better performance for medium or large runs, please cite
at least one of these six articles, in which results of non blocking
MPI runs are presented: \citet{PeKoLuMaLeCaLeMaLiBlNiBaTr11,KoErGoMi10,KoViCh10,Kom11,CaKoLaTiMiLeSnTr08,MaKoBlLe08}.\\


\noindent If you work on geophysical applications, you may be interested
in citing some of these application articles as well, among others:
\begin{description}
\item [{Southern California simulations}] ~\\
 \citet{KoLiTrSuStSh04,KrJiKoTr06a,KrJiKoTr06b}.


If you use the 3D southern California model, please cite \citet{SuSh03}
(Los Angeles model), \citet{lovelyetal06} (Salton Trough), and \citet{hauksson2000}
(southern California). The Moho map was determined by \citet{zhukanamori2000}.
The 1D SoCal model was developed by \citet{DrHe90}.

\item [{Anisotropy}] ~\\
 \citet{ChTr07,JiTsKoTr05,ChFaKo04,FaChKo04,RiRiKoTrHe02,TrKo00}.
\item [{Attenuation}] ~\\
 \citet{SaKoTr10,KoTr02a,KoTr99}.
\item [{Topography}] ~\\
 \citet{LeKoHuTr09,LeChKoHuTr09,LeChLiKoHuTr08,GoAmTaCaSmSaMaKo09,WiKoScTr04}.
\end{description}
The corresponding Bib\TeX{} entries may be found in file \texttt{doc/USER\_MANUAL/bibliography.bib}.


\section{Support}

This material is based upon work supported by the USA National Science
Foundation under Grants No. EAR-0406751 and EAR-0711177, by the French
CNRS, French INRIA Sud-Ouest MAGIQUE-3D, French ANR NUMASIS under
Grant No. ANR-05-CIGC-002, and European FP6 Marie Curie International
Reintegration Grant No. MIRG-CT-2005-017461. Any opinions, findings,
and conclusions or recommendations expressed in this material are
those of the authors and do not necessarily reflect the views of the
USA National Science Foundation, CNRS, INRIA, ANR or the European
Marie Curie program.

